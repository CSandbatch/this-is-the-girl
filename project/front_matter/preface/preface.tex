% preface
\cleardoublepage
\providecommand{\MatterHeading}[1]{\ifdefined\chapter\chapter*{#1}\else\section*{#1}\fi}

\MatterHeading{Preface}
\thispagestyle{empty}

\setlength{\parindent}{1.25em}
\setlength{\parskip}{0pt}

\par

\quad Certain emotional states command recurrence, others collapse at their moment of instance. From others still, we can manage little more than a disciplined withdrawal. What emerged from this second book (but where is the first?) was not a sequence of lyrical sallies but a chamber architecture, that is, three movements governed less by subject matter than by those subjects' inescapability.

\par\medskip

The organizing terms—\textit{Obsession}, \textit{Ruin}, \textit{Désir}—are therefore not psychological states so much as climates of tonality. Obsession is treated as a formal problem: repetition, fixation, excess of return. Ruin is not aftermath but atmosphere: the condition in which ethical and aesthetic structures decay at different speeds. Desire, finally, appears not as appetite but as residue; the irradiated substance remaining after fixation and collapse have exhausted themselves.

\par\medskip

\textit{What does that remainder look like?} My answer, and the answer I submit here, is \textit{music}. I have come to think of poems, especially in sequence, as inhabiting keys rather than themes. A key does not dictate content; it establishes gravity. It determines what kinds of motion feel natural, what kinds of resolution feel false, what gestures must be repeated until they either break or transfigure. In this sense, tonality becomes a discipline rather than a metaphor—a way of preventing mere intensity from mistaking itself for meaning.

\par\medskip

Tonality, as I use it here, is not nostalgia for harmony but a refusal of arbitrariness. Key supplies orientation without promising resolution. It permits dissonance, even insists upon it, but denies dissonance the privilege of pretending to be depth on its own. Under tonal constraint, excess must justify itself; repetition must prove endurance; collapse must leave a paw print worth tracking.

\par\medskip

To score a book in this way is to betray belief that tonal coherence still matters. It may seem antiquated, even reactionary. That charge I accept, but these poems confess nothing.
