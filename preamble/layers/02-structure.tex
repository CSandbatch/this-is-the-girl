% =========================================================
% LAYER 2 - STRUCTURE NODES (RECTO-AWARE)
% =========================================================
% @layer: 2
% @role: structure
% @depends_on: layer0, layer1
% @aesthetic: none
% @stability: stable
%
% PURPOSE:
% Define structural entry nodes (opus, body items) and
% enforce recto starts for all major divisions.
% =========================================================

\ifdefined\TexGraphDriver\else
  \errmessage{This file is a preamble layer and must not be compiled directly. Compile the project driver.}%
\fi

\makeatletter

% ---------------- Recto-clear primitive ----------------
\newcommand{\RectoClear}{%
  \clearpage
  \if@twoside
    \ifodd\value{page}\else
      \hbox{}%
      \thispagestyle{empty}%
      \newpage
    \fi
  \fi
}

% ---------------- Section-title spread control ----------------
% Ensures each section title begins on a recto page and is preceded by a blank
% verso page. If the preceding section ended on verso, this will insert an
% additional blank recto page so that a blank verso can precede the title.
\newcommand{\SectionTitleClear}{%
  \clearpage
  \if@twoside
    % If we're at an odd page now, the previous page was even (verso). Insert a
    % blank recto so the title can still be preceded by a blank verso.
    \ifodd\value{page}%
      \thispagestyle{empty}\hbox{}%
      \newpage
    \fi
    % Now we're on an even page: ship a blank verso and advance to recto.
    \thispagestyle{empty}\hbox{}%
    \newpage
  \fi
}

% ---------------- Inner/outer margin alignment helpers ----------------
% Outer margin: right on odd pages, left on even pages (twoside).
% Inner margin: left on odd pages, right on even pages (twoside).
\newcommand{\TexGraphOuterLine}[1]{%
  \noindent
  \if@twoside
    \ifodd\value{page}%
      \hfill\mbox{#1}%
    \else
      \mbox{#1}\hfill%
    \fi
  \else
    \hfill\mbox{#1}%
  \fi
  \par
}

\newcommand{\TexGraphInnerLine}[1]{%
  \noindent
  \if@twoside
    \ifodd\value{page}%
      \mbox{#1}\hfill%
    \else
      \hfill\mbox{#1}%
    \fi
  \else
    \mbox{#1}\hfill%
  \fi
  \par
}

% ---------------- Fit-to-measure helpers (single-line headings) ----------------
% These choose the largest size from a ladder that fits within \linewidth.
\newcommand{\TexGraphFitTitleText}[1]{%
  \begingroup
    \setbox0=\hbox{{\Large #1}}%
    \ifdim\wd0>\linewidth
      \setbox0=\hbox{{\large #1}}%
      \ifdim\wd0>\linewidth
        \setbox0=\hbox{{\normalsize #1}}%
        \ifdim\wd0>\linewidth
          \setbox0=\hbox{{\small #1}}%
        \fi
      \fi
    \fi
    \box0
  \endgroup
}

\newcommand{\TexGraphFitSubtitleText}[1]{%
  \begingroup
    \setbox0=\hbox{{\normalsize #1}}%
    \ifdim\wd0>\linewidth
      \setbox0=\hbox{{\small #1}}%
      \ifdim\wd0>\linewidth
        \setbox0=\hbox{{\footnotesize #1}}%
        \ifdim\wd0>\linewidth
          \setbox0=\hbox{{\scriptsize #1}}%
        \fi
      \fi
    \fi
    \box0
  \endgroup
}

% ---------------- Matter transitions begin recto ----------------
% Use a single, continuous page numbering scheme (no roman front-matter, no
% reset at \mainmatter). Keep \@mainmatter toggles for class semantics.
\renewcommand{\frontmatter}{%
  \RectoClear
  \@mainmatterfalse
}

\renewcommand{\mainmatter}{%
  \RectoClear
  \@mainmattertrue
}

\renewcommand{\backmatter}{%
  \RectoClear
  \@mainmatterfalse
}

\makeatother

% ---------------- Body item wrapper ----------------
% Items start on a new page, but do not force recto (avoids blank pages).
\newcommand{\BodyItem}[1]{%
  \clearpage
  \input{#1}%
}

% ---------------- Prose poem helpers (semantic, style-overrideable) ----------------
% Styles may override these to tune spacing/indentation, but the project should
% compile even when prose poems use these helpers.
\providecommand{\ProsePoemHead}[2]{\PoemHead{#1}{#2}}
\providecommand{\ProseInsetLines}[1]{%
  \par
  \begingroup
    \leftskip=2em
    \rightskip=2em
    \parindent=0pt
    \parskip=0pt
    \obeylines
    #1\par
  \endgroup
  \par
}

% ---------------- Cycle movement paging ----------------
% After a cycle's title block, movement 1 continues on the same page.
% Movements 2+ begin on their own pages.
\newif\ifTexGraphFirstMovement
\newcommand{\CycleBegin}{\TexGraphFirstMovementtrue}
\newcommand{\CycleMovement}[1]{%
  \ifTexGraphFirstMovement
    \TexGraphFirstMovementfalse
  \else
    \clearpage
  \fi
  \input{#1}%
}

% ---------------- Opus node ----------------
\newcounter{TexGraphOpus}
\renewcommand{\theTexGraphOpus}{\Roman{TexGraphOpus}}

\newcommand{\TexGraphCurrentOpusId}{}
\newcommand{\TexGraphCurrentOpusTitle}{}

\newcommand{\TexGraphSetOpusMeta}[2]{%
  \gdef\TexGraphCurrentOpusId{#1}%
  \gdef\TexGraphCurrentOpusTitle{#2}%
  \markboth{#2}{#2}%
}

% \Opus{<id>}{<title>}{<entry-file>}{<body>}
\newcommand{\Opus}[4]{%
  \SectionTitleClear
  \refstepcounter{TexGraphOpus}%
  \TexGraphSetOpusMeta{#1}{#2}%
  \label{op:#1}%
  \input{#3}%
  #4%
}
